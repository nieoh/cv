\documentclass[11pt,a4paper,unicode]{moderncv}

% moderncv themes
\moderncvtheme{classic} % optional argument are 'blue' (default), 'orange', 'green', 'red', 'purple', 'grey' and 'roman' (for roman fonts, instead of sans serif fonts)
\usepackage{lipsum}
% character encoding
\usepackage[utf8]{inputenc} % replace by the encoding you are using

% adjust the page margins
\usepackage[scale=0.85]{geometry}
% \setlength{\hintscolumnwidth}{3cm} % if you want to change the width of the column with the dates
% \AtBeginDocument{\setlength{\maketitlenamewidth}{6cm}} % only for the classic theme, if you want to change the width of your name placeholder (to leave more space for your address details
% \AtBeginDocument{\recomputelengths} % required when changes are made to page layout lengths

% Hyperlinks
\usepackage[unicode]{hyperref} % to use hyperlinks
\definecolor{linkcolour}{rgb}{0,0.2,0.6} % hyperlinks setup
\hypersetup{colorlinks,breaklinks,urlcolor=linkcolour, linkcolor=linkcolour}



\firstname{Stephanie} \familyname{Oh}

\address{Los Angeles, CA} \phone{224.603.2085} \email{oh.stephanie.s@gmail.com}
\homepage{www.linkedin.com/in/ohstephanie}
 

% to show numerical labels in the bibliography; only useful if you make citations in your resume
\makeatletter \renewcommand*{\bibliographyitemlabel}{\@biblabel{\arabic{enumiv}}} \makeatother

\nopagenumbers{}

\begin{document}

\vspace*{-9mm}
\maketitle
\vspace*{-9mm}
\section{Education}
\cventry{2013}{Masters of Science in Mathematics}{Purdue University}{West Lafayette,
  IN}{\textit{3.6/4.0}}{Recipient of fellowships for outstanding PhD-track students.}

\cventry{2011}{Bachelor of Arts in Mathematics with Honors}{Northwestern University}{Evanston,
  IL}{\textit{3.64/4.0}}{Thesis on Expander Graphs; Recipient of merit scholarships for outstanding
  academic achievement.} \cventry{2007}{High School Diploma}{The Hill School}{Pottstown, PA}{}{}

\cventry{2009-2010}{Budapest Semesters in Mathematics}{Budapesti M\H{u}szaki \'{e}s Gazdas\'{a}gtudom\'{a}nyi Egyetem}{Budapest, Hungary}{}{}

\section{Jobs}
\cventry{2013}{Developer}{Create and maintain math content for
  \url{wolframalpha.com} from simple arithmetic operations through college level calculus material. \newline \listitemsymbol Helped launch \url{wolframalpha.com/problem-generator} by testing and debugging the random problem generation feature and polishing language. \newline \listitemsymbol Work in a small team to conceive and develop tutorials for Wolfram Language -- a functional programming language used in Mathematica -- targeted towards first time programmers and hobbyists. \newline \listitemsymbol Use gamification and visual aid to produce creative methods of explaining programming and mathematical concepts}{}{}{} 

\cventry{2012-2013}{Teaching Assistant}{Administered and graded quizzes and conducted recitation for
  179 students in Calculus I and II at Purdue University}{}{}{}

\section{Professional Affiliations and Development}
\cvline{2013}{Attended the Wolfram Technology Conference.}{}{}{}{}

\cvline{2011-2013}{Member of American Mathematical Society.}{}{}{}{}

\cvline{2011}{Presented a poster on Expander graphs at the Joint Mathematics Meeting, the largest
  international math conference.}{}{}{}{}

\cvline{2010-2011}{Member of Mortar Board national honor society recognizing college seniors for
  their exemplary scholarship, leadership and service.}{}{}{}{}

\cvline{2010}{Undergraduate research program funded by an NSF grant studying Expander Graphs with professor Darren Long at the University of California, Santa Barbara.}

\cvline{2009}{Attended the Nebraska Conference for Undergraduate Women in Mathematics.}

\cvline{2008}{Selected for and attended Carleton College's Summer Math Program for Women.}{}{}{}{}

\section{Computer skills}
\cvline{Proficient}{Mathematica, Python}

\cvline{Basic}{Java, C}

\cvline{Miscellaneous}{Microsoft Office, Linux, \LaTeX}

%\section{Qualities}
%\cvline{Personal skills:}{Strong analytical, problem-solving and communication skills; eager to
%  learn and develop new skills; ability to design creative solutions.}  \cvline{Experience
%  with:}{Working in a small team to conceive and develop a new product; explaining technical details in understandable terms; %training students to achieve mathematical maturity (or should I say technical proficiency?). }

\section{Languages}
\cvline{Fluent}{English, Korean}{} \cvline{Basic}{French, German}{}

%\section{Extracurricular}
%\cvline{Interests:}{Traveling, playing the piano, cycling and cooking.}

%\cvline{Community service:}{Participated in a clinical trip providing free medical attention in southern
%  Vietnam.  \newline Tutored Korean K-12 students in math and English.}

%=====================================================
% Cover Letter
%=====================================================
\clearpage

\recipient{HR Departmnet}{Corporation\\123 Pleasant Lane\\12345 City, State} % Letter recipient
\date{\today} % Letter date
\opening{Dear Sir or Madam,} % Opening greeting
\closing{Sincerely yours,} % Closing phrase
\enclosure[Attached]{curriculum vit\ae{}} % List of enclosed documents

\makelettertitle % Print letter title

Going into math grad school, I thought I would stay in academia, in particular mathematics, forever. I soon realized grad school wasn't for me. I thought maybe it was the lack of real-life application so I got a job at Wolfram where I could still stay in touch with my mathematical knowledge, and not just the analytical skills I had developed in grad school, but applying it directly to math content management. Programming was just an added bonus. However, after a few months at Wolfram, I realized that the part of grad school I enjoyed wasn't just the mathematical knowledge and the part of grad school I disliked wasn't just the lack of application. 

Parts of grad school I enjoyed:
\\- Problem solving: Sitting down with a particular tough question, trying out different paths of solutions, asking my peers for help and brainstorming together to finally solve the problem.
\\- Having a dictionary of highly technical vocabulary: Plainly put, I like technical jargon. We make up words with specific definitions to make communication within the field easier. 
\\- Teaching: Not in contrast, but in queue with my previous point, I enjoy explaining technical details in layman's terms. Highly specialized vocabulary has its place when brainstorming with colleagues but once a solution has been found, relaying the information to others who weren't part of your original discussion or even a student who doesn't have the proper background, I find extremely rewarding.
\\- Colleagues: I enjoy being around knowledgeable and logical and ambitious and motivated people. 

Parts of grad school I didn't enjoy:
\\- Lack of opportunities to explore different subjects: Grad school is very specialized; I like to dabble in a variety of things. Very adaptible.
\\- Lack of application (my particular field): Rewarding to see real life results and feedback.
\\- Slow pace: I need something more stimulating. Grad school is very sedentary. I like a place with travel opportunity (adapt quickly to change in location), having to be "on-call" 24/7 in face of a deadline. I need to be able to deliver results.


Some place with the opportunity to grow.

\lipsum[1] % Dummy text

\makeletterclosing % Print letter signature


\end{document}